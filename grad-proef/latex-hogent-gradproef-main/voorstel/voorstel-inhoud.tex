%---------- Inleiding ---------------------------------------------------------

% TODO: Is dit voorstel gebaseerd op een paper van Research Methods die je
% vorig jaar hebt ingediend? Heb je daarbij eventueel samengewerkt met een
% andere student?
% Zo ja, haal dan de tekst hieronder uit commentaar en pas aan.

%\paragraph{Opmerking}

% Dit voorstel is gebaseerd op het onderzoeksvoorstel dat werd geschreven in het
% kader van het vak Research Methods dat ik (vorig/dit) academiejaar heb
% uitgewerkt (met medesturent VOORNAAM NAAM als mede-auteur).
% 

\section{Inleiding}%
\label{sec:inleiding}


Competitieve multiplayer platformers zijn een populair genre binnen de game-industrie. Deze spellen combineren behendigheid, strategie en snelle reflexen om een uitdagende en dynamische spelervaring te creëren. Met de opkomst van online gaming en esports groeit de vraag naar goed ontworpen multiplayer-ervaringen. Dit onderzoek richt zich op het ontwerpen en ontwikkelen van een competitieve multiplayer platformer in GameMaker Studio 2. Hierbij wordt gekeken naar game-mechanieken en spelbalans om een eerlijke en boeiende ervaring te bieden aan spelers.


\section{project beschrijving}%
\label{sec:literatuurstudie}

Dit project beoogt de ontwikkeling van een competitieve multiplayer platformer in GameMaker Studio 2. Het doel is om een spel te creëren waarin spelers het tegen elkaar kunnen opnemen in een dynamische en evenwichtige omgeving. Belangrijke aspecten die onderzocht worden, zijn:

Gameplay-mechanieken: Het implementeren van vloeiende bewegingen, sprongen en vallen om competitie te versterken.

Spelbalans: de eerste bij de finish wint.

Er zit functionalteit in Gamemaker studio 2 om een multiplayer game te realiseren.

Het onderzoek zal verschillende technische en ontwerpkeuzes analyseren en evalueren om te bepalen welke methoden het meest effectief zijn voor dit genre.
Hiermee wil ik bedoelen dat ik verschillende elemeneten zoals het movement systeem van super mario zal gebruiken. Uiteindelijk zal het project resulteren in een speelbare prototype dat getest wordt op gameplay, prestaties en spelerservaring.

% Voor literatuurverwijzingen zijn er twee belangrijke commando's:
% \autocite{KEY} => (Auteur, jaartal) Gebruik dit als de naam van de auteur
%   geen onderdeel is van de zin.
% \textcite{KEY} => Auteur (jaartal)  Gebruik dit als de auteursnaam wel een
%   functie heeft in de zin (bv. ``Uit onderzoek door Doll & Hill (1954) bleek
%   ...'')



